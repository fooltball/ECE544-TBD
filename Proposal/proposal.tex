\documentclass[10pt,twocolumn,letterpaper]{article}

\usepackage{cvpr}
\usepackage{times}
\usepackage{epsfig}
\usepackage{graphicx}
\usepackage{amsmath}
\usepackage{amssymb}

% Include other packages here, before hyperref.

% If you comment hyperref and then uncomment it, you should delete
% egpaper.aux before re-running latex.  (Or just hit 'q' on the first latex
% run, let it finish, and you should be clear).
\usepackage[breaklinks=true,bookmarks=false]{hyperref}

\cvprfinalcopy % *** Uncomment this line for the final submission

\def\cvprPaperID{****} % *** Enter the CVPR Paper ID here
\def\httilde{\mbox{\tt\raisebox{-.5ex}{\symbol{126}}}}

% Pages are numbered in submission mode, and unnumbered in camera-ready
%\ifcvprfinal\pagestyle{empty}\fi
\setcounter{page}{1}
\begin{document}

%%%%%%%%% TITLE
\title{Impact of Social Media on Stock Price}

\author{Kexin Hui, Dongbo Wang, Zhaoheng Hu\\
Department of Electrical and Computer Engineering\\
University of Illinois at Urbana-Champaign\\
%Institution1 address\\
{\tt\small \{khui3, dwang49, hu61\}@illinois.edu}
% For a paper whose authors are all at the same institution,
% omit the following lines up until the closing ``}''.
% Additional authors and addresses can be added with ``\and'',
% just like the second author.
% To save space, use either the email address or home page, not both
%\and
%Second Author\\
%Institution2\\
%First line of institution2 address\\
%{\tt\small secondauthor@i2.org}
}

\maketitle
%\thispagestyle{empty}

%%%%%%%%% ABSTRACT
\begin{abstract}
   This document serves as the proposal of our final project for ECE 544NA Fall 2017. Our team has three members and our project focusing on discovering the impact of social media on stock price.
\end{abstract}

%%%%%%%%% BODY TEXT
\section{Introduction}

There are always rumors that someone made a huge fortune overnight through investing on stock market, which draws almost every person’s interest trying to make money. Yet it is always hard to tell whether it is a good time to invest on stock market. Some may say that predicting the stock is much like flipping a coin at 50\% probability and the behaviors in the market are entirely random. Over years, researchers have shown
Nowadays, the influence of social media is undoutedly of particular importance. People like to post everything on social media platforms like Facebook and Twitter, not along to those great men who run a Fortune 500 company or are nominated as the World’s Billionaires. What they posted will more or less influence the company stock price they are related to. For example, on August 13th 2013, billionaire Carl Icahn, of Icahn Capital Management, announced his Apple position over Twitter with the accompanying comment of it being significantly undervalued. Within seconds Apple stock spiked and within minutes gained \$17 billion in market cap.  This cannot just be a coincidence.
Therefore, at this point, we want to explore more about the impacts of social media on stock market. The focus is on technology stocks, which attracts a lot of attention and exposure on the social media. We start from Tesla’s stock TSLA(NASDAQ) and tweets of Elon Musk, CEO of Tesla Inc.


%------------------------------------------------------------------------
\section{Proposed Method}

In the scope of this project, our main goal is to interpret the relationship between for example twitter posts and the corresponding stock price a certain time after the posts were made. We propose the following methods to handle certain tasks for the project.

%-------------------------------------------------------------------------
\subsection{Signal Processing}

The stock price \cite{StockPrice1} of certain company is basically a 1-dimensional signal in time domain. To extract features out of this signal, we might want to apply some simple signal processing method to make the trend more clear or reduce some random patterns in between.

%-------------------------------------------------------------------------
\subsection{NLP}

The task of extracting features out of the posts on social media falls naturally into the topics of Neuro-linguistic programming (NLP) \cite{NLP}.
We will apply some simple model or methods from NLP to extract meaningful features from the posts.

%-------------------------------------------------------------------------
\subsection{Deep Learning}

After the features are extracted, the majority of the learning work will be done be deep learning. We planned to build some neural network \cite{NN}, train and tune it and see if it is capable of learning the subtle relationship between posts and stock price.

%-------------------------------------------------------------------------
\section{Dataset Description}

To dig into the impacts of social media on stock market, we will use Elon Musk Tweets dataset, and Tesla Stock Price dataset from Kaggle. As is known to all, Elon Musk is the CEO of Tesla Inc. and what he tweeted everyday may more or less pose some influence on the stock price of Tesla. As a consequence, it's meaningful for us to figure out the structure and content of each dataset so we can fully make use of them to train a proper model which can tell us the impact of Elon Musk's tweets on Tesla's stock price.

After carefully reading the explanation and example of datasets, we have a comprehensive understanding about these datasets.

The first one is Elon Musk's tweets in past 7 years. It contains text and relevant information of all tweets which are posted by Elon Musk, CEO of Tesla, during 2010 to 2017. Mainly, 3 features will be used in our project:\\
1. Tweet id: representing the unique id of each tweet, it contains tweet-stamp.\\
2. Timestamp: the date and time of the day that corresponding tweet was posted, each record is in 24hr representation format.\\
3. Tweet text: the text of tweets, additionally, ‘b’ is removed.\\

Secondly, Tesla’s stock price in past 7 years is another essential dataset in the project. This dataset is telling us the historical price start from 2010, the IPO (initial public offering) year of the incorporation, to early 2017. There are several important features provided:\\
1. Date: the date of each price record.\\
2. Open:  the opening price of the stock.\\
3. High: the high price of that day.\\
4. Low: the low price of that day.\\
5. Close: the closed price of that day.\\
6. Volume: The amount of stocks traded during that day.\\
7. Adj close: The stock’s closing price that has been amended to include any distributions/corporate actions that occurs before next day open.

%-------------------------------------------------------------------------
\section{Proposed Experiments}
Currently, since the available data resources are Elon Musk’s tweets and historical stock prices from 2010 to 2017. We will choose first five years data (2010 - 2015) to train the model and use data of 2016 for validation. Finally, data generated in 2017 will be applied for testing.
For future work, if we have enough time and could find other incorporations’ CEOs’ posts and stock prices, we will perform training by using whole dataset of Elon Musk and Tesla to get a model and apply this model to other incorporations and their CEOs to test. In this way, we will probably be able to find the connection between CEO’s social media activities and stock price of the company.

%------------------------------------------------------------------------
\section{Resource Feasibility}

The following data are available:\\
1. On twitter posts: \url{https://www.kaggle.com/kingburrito666/elon-musk-tweets}\\
2. On stock price: \url{https://www.kaggle.com/rpaguirre/tesla-stock-price}

The data for certain stock price is easily reachable, either from multiple dataset available through Kaggle, or could be obtained from Google stock api.
The text data for twitter should be reachable as well. There are tools available on the internet, but we need to investigate further more and be careful no to break any regulations.

%------------------------------------------------------------------------
\section{Tentative timeline and the necessary steps}
We propose the tentative timeline (approximately 11 weeks) and necessary steps for this project, as shown in table 1.

\begin{table}
\begin{center}
\begin{tabular}{|l|c|}
\hline
Weeks & Steps \\
\hline\hline
$1\sim2$ & Data reading and cleaning \\
$2\sim3$ & Feature Extraction via NLP and DSP\\
$2\sim3$ & Apply Deep Learning and Training\\
$1\sim2$ & Validation and Testing\\
$0\sim1$ & Finalization and Documentation \\
\hline
\end{tabular}
\end{center}
\caption{Tentative timeline and necessary steps}
\end{table}

{\small
\bibliographystyle{ieee}
\bibliography{egbib}
}

\end{document}
